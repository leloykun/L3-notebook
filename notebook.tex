  \documentclass[8pt,a4paper,landscape,oneside]{amsart}
\usepackage{amsmath, amsthm, amssymb, amsfonts}
\usepackage[T1]{fontenc}
\usepackage[utf8]{inputenc}
\usepackage{booktabs}
\usepackage{caption}
\usepackage{fancyhdr}
\usepackage{float}
\usepackage{fullpage}
%\usepackage{geometry}
% \usepackage[top=0pt, bottom=1cm, left=0.3cm, right=0.3cm]{geometry}
\usepackage[top=3pt, bottom=1cm, left=0.3cm, right=0.3cm]{geometry}
\usepackage{graphicx}
% \usepackage{listings}
\usepackage{subcaption}
\usepackage[scaled]{beramono}
\usepackage{titling}
\usepackage{datetime}
\usepackage{enumitem}
\usepackage{multicol}
\usepackage{bookmark}
\usepackage{color}
\usepackage{xcolor}
\usepackage{soul}

\setcounter{tocdepth}{3}

\newcommand{\subtitle}[1]{%
  \posttitle{%
    \par\end{center}
    \begin{center}\large#1\end{center}
    \vskip0.1em\vspace{-1em}}%
}

% Minted
\usepackage{minted}
\newcommand{\code}[1]{\inputminted[fontsize=\normalsize,baselinestretch=1]{cpp}{_code/#1}}
\newcommand{\bashcode}[1]{\inputminted{bash}{_code/#1}}
\newcommand{\regcode}[1]{\inputminted{cpp}{code/#1}}

% Header/Footer
% \geometry{includeheadfoot}
%\fancyhf{}
\pagestyle{fancy}
\lhead{Ateneo de Manila University}
\rhead{\thepage}
\cfoot{}
\setlength{\headheight}{15.2pt}
\setlength{\droptitle}{-20pt}
\posttitle{\par\end{center}}
\renewcommand{\headrulewidth}{0.4pt}
\renewcommand{\footrulewidth}{0.4pt}

% Math and bit operators
\DeclareMathOperator{\lcm}{lcm}
\newcommand*\BitAnd{\mathrel{\&}}
\newcommand*\BitOr{\mathrel{|}}
\newcommand*\ShiftLeft{\ll}
\newcommand*\ShiftRight{\gg}
\newcommand*\BitNeg{\ensuremath{\mathord{\sim}}}
\DeclareRobustCommand{\stirling}{\genfrac\{\}{0pt}{}}

\newcommand{\sectionRed}[1]{\section{\colorbox{red}{\color{white}#1}}}
\newcommand{\subsectionRed}[1]{\subsection{\colorbox{red}{\color{white}#1}}}
\newcommand{\subsubsectionRed}[1]{\subsubsection{\colorbox{red}{\color{white}#1}}}

\newcommand{\sectionBlack}[1]{\section{\colorbox{black}{\color{white}#1}}}
\newcommand{\subsectionBlack}[1]{\subsection{\colorbox{black}{\color{white}#1}}}
\newcommand{\subsubsectionBlack}[1]{\subsubsection{\colorbox{black}{\color{white}#1}}}

\newenvironment{myitemize}
{ \begin{itemize}[leftmargin=.5cm]
    \setlength{\itemsep}{0pt}
    \setlength{\parskip}{0pt}
    \setlength{\parsep}{0pt}     }
{ \end{itemize}                  }

% Title/Author
\title{$AdMU Progvar$}
\subtitle{Team Notebook}
\date{\ddmmyyyydate{\today{}}}

% Output Verbosity
\newif\ifverbose
\verbosetrue
% \verbosefalse

\begin{document}

\begin{multicols*}{3}
\maketitle
\thispagestyle{fancy}
\vspace{-3em}
% \addtocontents{toc}{\protect\enlargethispage{\baselineskip}}
\tableofcontents

% \clearpage

\section{Code Templates}
	\code{header.cpp}
\section{Data Structures}
  \subsection{Union Find}
    \code{data-structures/union_find.cpp}
  \subsection{Fenwick Tree}
		\subsubsection{Fenwick Tree w/ Point Queries}
			\code{data-structures/fenwick.cpp}
    \subsubsection{Fenwick Tree w/ Max Queries}
      \code{data-structures/fenwick_max.cpp}
  \subsection{Segment Tree}
    \subsubsection{Recursive, Point-update Segment Tree}
      \code{data-structures/segtree_rec_PU_RQ.cpp}
    \subsubsection{Iterative, Point-update Segment Tree}
      \code{data-structures/segtree_iter_PU_RQ.cpp}
    \subsubsection{Pointer-based, Range-update Segment Tree}
      \code{data-structures/segtree_rec_RU_RQ.cpp}
    \subsubsection{Array-based, Range-update Segment Tree}
      \code{data-structures/segtree_array_lazy.cpp}
    \subsubsection{Array-based, Point-update, Persistent Segment Tree}
      \code{data-structures/segtree_persistent_iter.cpp}
    \subsubsection{Pointer-based, Point-update, Persistent Segment Tree}
      \code{data-structures/segtree_persistent_rec.cpp}
    \subsubsection{2D Segment Tree}
      \code{data-structures/segtree_2d.cpp}
  \subsection{Treap}
		\subsubsection{Implicit Treap}
			\code{data-structures/treap_implicit.cpp}
		\subsubsectionBlack{Persistent Treap}
  \subsectionRed{Splay Tree}
    \code{data-structures/splay.cpp}
	\subsection{Ordered Statistics Tree}
    \code{data-structures/ordered_statistics_tree.cpp}
  \subsection{Sparse Table}
    \subsubsection{1D Sparse Table}
      \code{data-structures/sparse.cpp}
    \subsubsectionRed{2D Sparse Table}
      \code{data-structures/sparse_2d.cpp}
  \subsectionRed{Misof Tree}
    A simple tree data structure for inserting, erasing, and querying the nth largest element.
    \code{data-structures/misof_tree.cpp}
\section{Graphs}
	Using adjacency list:
	\code{graphs/graph_template_adjlist.cpp}
	Using adjacency matrix:
	\code{graphs/graph_template_adjmat.cpp}
	Using edge list:
	\code{graphs/graph_template_edgelist.cpp}
	\subsection{Single-Source Shortest Paths}
		\subsubsection{Dijkstra}
			\code{graphs/shortest_paths/dijkstra.cpp}
		\subsubsection{Bellman-Ford}
			\code{graphs/shortest_paths/bellman_ford.cpp}
    \subsubsection{Shortest Path Faster Algorithm}
      \code{graphs/shortest_paths/spfa.cpp}
	\subsection{All-Pairs Shortest Paths}
		\subsubsection{Floyd-Washall}
			\code{graphs/shortest_paths/floyd_warshall.cpp}
	\subsection{Strongly Connected Components}
		\subsubsection{Kosaraju}
      \code{graphs/scc/kosaraju.cpp}
    \subsubsectionRed{Tarjan's Offline Algorithm}
      \code{graphs/scc/tarjan.cpp}
  \subsectionRed{Minimum Mean Weight Cycle}
    Run this for each strongly connected component
    \code{graphs/min_mean_cycle.cpp}
	\subsection{Biconnected Components}
	  \subsubsection{Cut Points, Bridges, and Block-Cut Tree}
      \code{graphs/bridges_artics.cpp}
		\subsubsection{Bridge Tree}
      Run the bridge finding algorithm first, burn the bridges, compress the
      remaining biconnected components, and then connect them using the bridges.
	\subsection{Minimum Spanning Tree}
		\subsubsection{Kruskal}
      \code{graphs/mst/kruskal.cpp}
		\subsubsection{Prim}
      \code{graphs/mst/prim.cpp}
	\subsectionRed{Euler Path/Cycle}
    \subsubsectionRed{Euler Path/Cycle in a Directed Graph}
      \code{graphs/euler_path.cpp}
    \subsubsectionRed{Euler Path/Cycle in an Undirected Graph}
      \code{graphs/euler_path_undirected.cpp}
	\subsectionRed{Bipartite Matching}
		\subsubsectionRed{Alternating Paths Algorithm}
      \code{graphs/bipartite_matching/bipartite_matching.cpp}
		\subsubsectionRed{Hopcroft-Karp Algorithm}
      \code{graphs/bipartite_matching/hopcroft_karp.cpp}
    \subsubsectionRed{Minimum Vertex Cover in Bipartite Graphs}
      \code{graphs/bipartite_matching/bipartite_mvc.cpp}
	\subsection{Maximum Flow}
		\subsubsection{Edmonds-Karp}
			\code{graphs/max_flow/edmonds_karp.cpp}
		\subsubsection{Dinic}
			\code{graphs/max_flow/dinic.cpp}
  \subsection{Minimum Cost Maximum Flow}
    \code{graphs/max_flow/mcst.cpp}
  \subsectionRed{All-pairs Maximum Flow}
    \subsubsectionRed{Gomory-Hu}
      \code{graphs/max_flow/gomory_hu_tree.cpp}
  \subsectionRed{Minimum Arborescence}
    Given a weighted directed graph, finds a subset of edges of minimum
    total weight so that there is a unique path from the root $r$ to each
    vertex. Returns a vector of size $n$, where the $i$th element is the
    edge for the $i$th vertex. The answer for the root is undefined!
    \code{graphs/arborescence.cpp}
  \subsectionRed{Blossom algorithm}
    Finds a maximum matching in an arbitrary graph in $O(|V|^4)$ time. Be
    vary of loop edges.
    \code{graphs/blossom.cpp}
  \subsection{Maximum Density Subgraph}
    Given (weighted) undirected graph $G$. Binary search density. If $g$ is
    current density, construct flow network: $(S, u, m)$, $(u, T,
    m+2g-d_u)$, $(u,v,1)$, where $m$ is a large constant (larger than sum
    of edge weights). Run floating-point max-flow. If minimum cut has empty
    $S$-component, then maximum density is smaller than $g$, otherwise it's
    larger. Distance between valid densities is at least $1/(n(n-1))$. Edge
    case when density is $0$. This also works for weighted graphs by
    replacing $d_u$ by the weighted degree, and doing more iterations (if
    weights are not integers).
  \subsection{Maximum-Weight Closure}
    Given a vertex-weighted directed graph $G$. Turn the graph into a flow
    network, adding weight $\infty$ to each edge. Add vertices $S,T$. For
    each vertex $v$ of weight $w$, add edge $(S,v,w)$ if $w\geq 0$, or edge
    $(v,T,-w)$ if $w<0$. Sum of positive weights minus minimum $S-T$ cut is
    the answer. Vertices reachable from $S$ are in the closure. The
    maximum-weight closure is the same as the complement of the
    minimum-weight closure on the graph with edges reversed.
  \subsection{Maximum Weighted Ind. Set in a Bipartite Graph}
    This is the same as the minimum weighted vertex cover. Solve this by
    constructing a flow network with edges $(S,u,w(u))$ for $u\in L$,
    $(v,T,w(v))$ for $v\in R$ and $(u,v,\infty)$ for $(u,v)\in E$. The
    minimum $S,T$-cut is the answer. Vertices adjacent to a cut edge are
    in the vertex cover.
  \subsection{Synchronizing word problem}
    A DFA has a synchronizing word (an input sequence that moves all states
    to the same state) iff.\ each pair of states has a synchronizing word.
    That can be checked using reverse DFS over pairs of states. Finding the
    shortest synchronizing word is NP-complete.
  \subsection{Max flow with lower bounds on edges}
    % TODO: Test this!
    Change edge $(u,v,l\leq f\leq c)$ to $(u,v,f\leq c-l)$. Add edge
    $(t,s,\infty)$. Create super-nodes $S$, $T$. Let $M(u) = \sum_{v}
    l(v,u) - \sum_{v} l(u,v)$. If $M(u)<0$, add edge $(u,T,-M(u))$, else
    add edge $(S,u,M(u))$. Max flow from $S$ to $T$. If all edges from $S$
    are saturated, then we have a feasible flow. Continue running max flow
    from $s$ to $t$ in original graph.
    % TODO: Was there something similar for vertex capacities that we should add?
  \subsection{Tutte matrix for general matching}
    Create an $n\times n$ matrix $A$. For each edge $(i,j)$, $i<j$, let
    $A_{ij} = x_{ij}$ and $A_{ji} = -x_{ij}$. All other entries are $0$.
    The determinant of $A$ is zero iff.\ the graph has a perfect matching.
    A randomized algorithm uses the Schwartz--Zippel lemma to check if it is
    zero.
  \subsection{Heavy Light Decomposition}
    \code{graphs/heavy_light_decomposition.cpp}
	\subsectionRed{Centroid Decomposition}
    \code{graphs/centroid_decomposition.cpp}
	\subsection{Least Common Ancestor}
		\subsubsection{Binary Lifting}
      \code{graphs/lca/binary_lifting.cpp}
    \subsubsectionBlack{Euler Tour Sparse Table}
    \subsubsectionBlack{Tarjan Off-line LCA}
  \subsection{Counting Spanning Trees}
    Kirchoff's Theorem: The number of spanning trees of any graph is the
    determinant of any cofactor of the Laplacian matrix in $O(n^3)$.
    \begin{enumerate}
        \item Let $A$ be the adjacency matrix.
        \item Let $D$ be the degree matrix (matrix with vertex degrees on the diagonal).
        \item Get $D-A$ and delete exactly one row and column. Any row and
        column will do. This will be the cofactor matrix.
        \item Get the determinant of this cofactor matrix using Gauss-Jordan.
        \item $\text{Spanning Trees} = \left|\mathrm{cofactor}(D-A) \right|$
    \end{enumerate}
  \subsection{Erd\H{o}s-Gallai Theorem}
    A sequence of non-negative integers $d_1 \ge \cdots \ge d_n$ can be represented as the
    degree sequence of finite simple graph on $n$ vertices if and only if $d_1 + \cdots + d_n$ is
    even and the following holds for $1 \le k \le n$:
    \[
    \sum_{i=1}^n d_i \le k(k-1) + \sum_{i=k+1}^n \min\left(d_i, k\right)
    \]
  \subsectionRed{Tree Isomorphism}
    \code{graphs/tree_isomorphism.cpp}
\section{Strings}
  \subsectionRed{Knuth-Morris-Pratt}
    Count and find all matches of string $f$ in string $s$ in $O(n)$ time.
    \code{strings/kmp.cpp}
  \subsection{Trie}
    \code{strings/trie.cpp}
    \subsubsection{Persistent Trie}
      \code{strings/trie_persistent.cpp}
  \subsectionRed{Suffix Array}
    Construct a sorted catalog of all substrings of $s$ in $O(n \log n)$ time using counting sort.
    \code{strings/suffix-array.cpp}
  \subsectionRed{Longest Common Prefix}
    Find the length of the longest common prefix for every substring in $O(n)$.
    \code{strings/lcp.cpp}
  \subsectionRed{Aho-Corasick Trie}
    Find all multiple pattern matches in $O(n)$ time. This is KMP for multiple strings.
    \code{strings/aho-corasick-trie.java}
  \subsection{Palimdromes}
    \subsubsectionRed{Palindromic Tree}
      Find lengths and frequencies of all palindromic substrings of a string in $O(n)$ time.

      Theorem: there can only be up to $n$ unique palindromic substrings for any string.
      \code{strings/palindromic-tree.cpp}
    \subsubsectionBlack{Eertree}
  \subsectionRed{Z Algorithm}
    Find the longest common prefix of all substrings of $s$ with itself in $O(n)$ time.
    \code{strings/z.cpp}
  \subsectionRed{Booth's Minimum String Rotation}
    Booth's Algo: Find the index of the lexicographically least string rotation in $O(n)$ time.
    \code{strings/booth.cpp}
	\subsection{Hashing}
    \subsubsection{Rolling Hash}
      \code{strings/rolling_hash.cpp}
\section{Number Theory}
	\subsection{Eratosthenes Prime Sieve}
    \code{numtheory/prime-sieve.cpp}
  \subsection{Divisor Sieve}
    \code{numtheory/divisor-sieve.cpp}
  \subsection{Number/Sum of Divisors}
    If a number $n$ is prime factorized where $n = {p_1}^{e_1} \times {p_2}^{e_2} \times \cdots \times {p_k}^{e_k}$, where $\sigma_0$ is the number of divisors while $\sigma_1$ is the sum of divisors:
    \[
    \sum_{d\mid n} d^k = \sigma_k (n) = \prod \frac{{p_i}^{k(e_i)+1}-1}{p_i -1}
    \]
    \[
    \text{Product: } \prod_{d\mid n} d = n^{\frac{\sigma_1 (n)}{2}}
    \]
  \subsection{M\"{o}bius Sieve}
    The M\"{o}bius function $\mu$ is the M\"{o}bius inverse of $e$ such that $e(n) = \sum_{d\mid n} \mu(d)$.
    \code{numtheory/moebius-sieve.cpp}
  \subsection{M\"{o}bius Inversion}
    Given arithmetic functions $f$ and $g$:
    \[
    g(n) = \sum_{d\mid n} f(d) \quad \Leftrightarrow \quad f(n) = \sum_{d\mid n} \mu(d)\; g\left(\frac{n}{d}\right)
    \]
  \subsection{GCD Subset Counting}
    Count number of subsets $S \subseteq A$ such that $\gcd(S) = g$ (modifiable).
    \code{numtheory/gcd-subsets.cpp}
  \subsection{Euler Totient}
    Counts all integers from 1 to $n$ that are relatively prime to $n$ in $O(\sqrt{n})$ time.
    \code{numtheory/totient.cpp}
  \subsection{Euler Phi Sieve}
    Sieve version of Euler totient, runs in $O(N \log N)$ time. Note that $n = \sum_{d\mid n} \varphi(d)$.
    \code{numtheory/phi-sieve.cpp}
  \subsection{Extended Euclidean}
    Assigns $x,y$ such that $ax + by = \gcd(a,b)$ and returns $\gcd(a,b)$.
    \code{numtheory/extended-euclidean.cpp}
  \subsection{Modular Exponentiation}
    Find $b^e \pmod m$ in $O(log e)$ time.
    \code{numtheory/mod_pow.cpp}
  \subsection{Modular Inverse}
    Find unique $x$ such that $ax \equiv 1 \pmod m$. Returns 0 if no unique solution is found. \underline{Please use modulo solver for the non-unique case.}
    \code{numtheory/modinv.cpp}
  \subsection{Modulo Solver}
    Solve for values of $x$ for $ax \equiv b \pmod m$. Returns $(-1,-1)$ if there is no solution. Returns a pair $(x, M)$ where solution is $x \bmod M$.
    \code{numtheory/modsolver.cpp}
  \subsection{Linear Diophantine}
    Computes integers $x$ and $y$ such that $ax+by=c$, returns $(-1,-1)$ if no solution. \underline{Tries to return positive integer answers for $x$ and $y$ if possible.}
    \code{numtheory/linear-diophantine.cpp}
  \subsection{Chinese Remainder Theorem}
    Solves linear congruence $x \equiv b_i \pmod {m_i}$. Returns $(-1,-1)$ if there is no solution. Returns a pair $(x, M)$ where solution is $x \bmod M$.
    \code{numtheory/chinese-remainder.cpp}
    \subsubsection{Super Chinese Remainder}
      Solves linear congruence $a_i x \equiv b_i \pmod {m_i}$. Returns $(-1,-1)$ if there is no solution.
      \code{numtheory/super-crt.cpp}
  \subsection{Primitive Root}
    \code{numtheory/primitive_root.cpp}
  \subsection{Josephus}
    Last man standing out of $n$ if every $kth$ is killed. Zero-based, and does not kill $0$ on first pass.
    \code{numtheory/josephus.cpp}
  \subsection{Number of Integer Points under a Lines}
    Count the number of integer solutions to $Ax+By \leq C$, $0 \leq x \leq n$,
    $0 \leq y$. In other words, evaluate the sum $\sum_{x=0}^n\left\lfloor\dfrac{C-Ax}{B}+1\right\rfloor$.
    To count all solutions, let $n = \left\lfloor\dfrac{c}{a}\right\rfloor$.
    In any case, it must hold that $C-nA \geq 0$. Be very careful about overflows.
\section{Algebra}
  \subsection{Fast Fourier Transform}
    Compute the Discrete Fourier Transform (DFT) of a polynomial in $O(n \log n)$ time.
    \code{algebra/fft.cpp}
  \subsection{FFT Polynomial Multiplication}
    Multiply integer polynomials $a, b$ of size $an, bn$ using FFT in $O(n \log n)$. Stores answer in an array $c$, rounded to the nearest integer (or double).
    \code{algebra/fft-poly-mul.cpp}
  \subsection{Number Theoretic Transform}
    Other possible moduli: $2 113 929 217 (2^{25}), 2 013 265 920 268 435 457 (2^{28}, with g = 5)$
    \code{algebra/ntt.cpp}
  \subsection{Polynomial Long Division}
    Divide two polynomials $A$ and $B$ to get $Q$ and $R$, where $\frac{A}{B} = Q + \frac{R}{B}$.
    \code{algebra/poly-long-div.cpp}
  \subsection{Matrix Multiplication}
    Multiplies matrices $A_{p\times q}$ and $B_{q\times r}$ in $O(n^3)$ time, modulo \texttt{MOD}.
    \code{algebra/matmul.java}
  \subsection{Matrix Power}
    Computes for $B^e$ in $O(n^3 \log e)$ time. Refer to Matrix Multiplication.
    \code{algebra/matpow.java}
  \subsection{Fibonacci Matrix}
    Fast computation for $n$th Fibonacci $\left\{F_1,F_2,\ldots,F_n \right\}$ in $O(\log n)$:
    \[
    \begin{bmatrix}
        F_n \\
        F_{n-1}
    \end{bmatrix}
    =
    \begin{bmatrix}
        1 & 1 \\
        1 & 0
    \end{bmatrix}^n
    \times
    \begin{bmatrix}
        F_2 \\
        F_1
    \end{bmatrix}
    \]
  \subsection{Gauss-Jordan/Matrix Determinant}
    Row reduce matrix $A$ in $O(n^3)$ time. Returns \texttt{true} if a solution exists.
    \code{algebra/gauss-jordan.java}
\section{Combinatorics}
  \subsection{Lucas Theorem}
    Compute $\binom{n}{k} \bmod{p}$ in $O(p + \log_p n)$ time, where $p$ is a prime.
    \code{combs/lucas.cpp}
  \subsection{Granville's Theorem}
    Compute $\binom{n}{k} \bmod{m}$ (for any $m$) in $O(m^2 \log^2 n)$ time.
    \code{combs/granville.py}
  \subsection{Derangements}
    Compute the number of permutations with $n$ elements such that no element is at their original position:
    \[
    D(n) = (n-1) \left( D(n-1) + D(n-2) \right) =  n D(n-1) + (-1)^n
    \]
    % \subsection{Next Permutation}
    % \subsection{Next Combination}
    % \subsection{Next Arrangement}
    % \subsection{Next Partition}
    \subsection{Factoradics}
    Convert a permutation of $n$ items to factoradics and vice versa in $O(n \log n)$.
    \code{combs/factoradics.cpp}
  \subsection{$k$th Permutation}
    Get the next $k$th permutation of $n$ items, if exists, using factoradics. All values should be from $0$ to $n-1$. Use factoradics methods as discussed above.
    \code{combs/kth-permutation.cpp}
  \subsection{Catalan Numbers}
    \[
        C_n = \frac{1}{n+1}\binom{2n}{n} =  \binom{2n}{n}-\binom{2n}{n+1}
    \]
    \begin{enumerate}
      \item The number of non-crossing partitions of an $n$-element set
      \item The number of expressions with $n$ pairs of parentheses
      \item The number of ways $n+1$ factors can be parenthesized
      \item The number of full binary trees with $n+1$ leaves
      \item The number of monotonic lattice paths of an $n \times n$ grid (5-SAT problem)
      \item The number of triangulations of a convex polygon with $n+2$ sides (non-rotational)
      \item The number of permutations $\{1, \ldots, n\}$ without a 3-term increasing subsequence
      \item The number of ways to form a mountain range with $n$ ups and $n$ downs
    \end{enumerate}
  \subsection{Stirling Numbers}
    $s_1$: Count the number of permutations of $n$ elements with $k$ disjoint cycles

    $s_2$: Count the ways to partition a set of $n$ elements into $k$ nonempty subsets
    \[
    s_1(n,k) = \begin{cases}
        1 & n = k = 0 \\
        s_1(n-1,k-1) - (n-1) s_1(n-1,k) & n,k>0 \\
        0 & \text{elsewhere}
    \end{cases}
    \]
    \[
    s_2(n,k) = \begin{cases}
        1 & n = k = 0 \\
        s_2(n-1,k-1) + k s_2(n-1,k) & n,k>0 \\
        0 & \text{elsewhere}
    \end{cases}
    \]
  \subsection{Partition Function}
    Pregenerate the number of partitions of positive integer $n$ with $n$ positive addends.
    \[
    p(n,k) = \begin{cases}
        1 & n = k = 0 \\
        0 & n < k \\
        p(n-1,k-1) + p(n-k,k) & n \ge k
    \end{cases}
    \]
\section{Geometry}
  \code{geom/compgeom.cpp}
	\subsection{Dots and Cross Products}
    \code{geom/dot-cross.cpp}
  \subsection{Angles and Rotations}
    \code{geom/angles-rots.cpp}
	\subsection{Spherical Coordinates}
    \[
        \begin{array}{cc}
            x = r \cos \theta \cos \phi & r = \sqrt{x^2 + y^2 + z^2} \\
            y = r \cos \theta \sin \phi & \theta = \cos^{-1} x/r \\
            z = r \sin \theta & \phi = \mathrm{atan2}(y,x)
        \end{array}
    \]
	\subsection{Point Projection}
    \code{geom/pt-proj.cpp}
	\subsection{Great Circle Distance}
    \code{geom/great-circle.cpp}
	\subsection{Point/Line/Plane Distances}
    \code{geom/dists.cpp}
	\subsection{Intersections}
    \subsubsection{Line-Segment Intersection}
      Get intersection points of 2D lines/segments $\overline{ab}$ and $\overline{cd}$.
      \code{geom/line-seg-isect.cpp}
    \subsubsection{Circle-Line Intersection}
      Get intersection points of circle at center $c$, radius $r$, and line $\overline{ab}$.
      \code{geom/circ-line-isect.cpp}
    \subsubsection{Circle-Circle Intersection}
      \code{geom/circ-circ-isect.cpp}
	\subsection{Polygon Areas}
    Find the area of any 2D polygon given as points in $O(n)$.
    \code{geom/poly-area.cpp}
    \subsubsection{Triangle Area}
      Find the area of a triangle using only their lengths. Lengths must be valid.
      \code{geom/tri-area.cpp}
    \subsubsection*{Cyclic Quadrilateral Area}
      Find the area of a cyclic quadrilateral using only their lengths. A quadrilateral is
      cyclic if its inner angles sum up to $360^\circ$.
      \code{geom/cyc-quad-area.cpp}
  \subsection{Polygon Centroid}
    Get the centroid/center of mass of a polygon in $O(m)$.
    \code{geom/poly-centroid.cpp}
  \subsection{Convex Hull}
    Get the convex hull of a set of points using Graham-Andrew's scan. This sorts the
    points at $O(n \log n)$, then performs the Monotonic Chain Algorithm at $O(n)$.
    \code{geom/convex-hull.cpp}
  \subsection{Point in Polygon}
    Check if a point is strictly inside (or on the border) of a polygon in $O(n)$.
    \code{geom/pt-in-poly.cpp}
  \subsection{Cut Polygon by a Line}
    Cut polygon by line $\overline{ab}$ to its left in $O(n)$, such that $\angle abp$ is counter-clockwise.
    \code{geom/cut-poly.cpp}
  \subsection{Triangle Centers}
    \code{geom/tri-centers.cpp}
  \subsection{Convex Polygon Intersection}
    Get the intersection of two convex polygons in $O(n^2)$.
    \code{geom/convex-poly-isect.cpp}
  \subsection{Pick's Theorem for Lattice Points}
    Count points with integer coordinates inside and on the boundary of a polygon in
    $O(n)$ using Pick's theorem: $\text{Area} = I + B/2 - 1$.
    \code{geom/picks.cpp}
  \subsection{Minimum Enclosing Circle}
    Get the minimum bounding ball that encloses a set of points (2D or 3D) in $\Theta{n}$.
    \code{geom/min-enclosing-circ.cpp}
  \subsection{Shamos Algorithm}
    Solve for the polygon diameter in $O(n \log n)$.
    \code{geom/shamos.cpp}
  \subsection{$k$D Tree}
    Get the $k$-nearest neighbors of a point within pruned radius in $O(k \log k \log n)$.
    \code{geom/kd-tree.cpp}
  \subsection{Line Sweep (Closest Pair)}
    Get the closest pair distance of a set of points in $O(n \log n)$ by sweeping a line and
    keeping a bounded rectangle. Modifiable for other metrics such as Minkowski and
    Manhattan distance. For external point queries, see $k$D Tree.
    \code{geom/closest-pair.cpp}
  \subsection{Line upper/lower envelope}
    To find the upper/lower envelope of a collection of lines $a_i+b_i x$,
    plot the points $(b_i,a_i)$, add the point $(0,\pm \infty)$ (depending
    on if upper/lower envelope is desired), and then find the convex hull.
  \subsection{Formulas}
    Let $a = (a_x, a_y)$ and $b = (b_x, b_y)$ be two-dimensional vectors.
    \begin{itemize}
      \item $a\cdot b = |a||b|\cos{\theta}$, where $\theta$ is the angle
        between $a$ and $b$.
      \item $a\times b = |a||b|\sin{\theta}$, where $\theta$ is the
        signed angle between $a$ and $b$.
      \item $a\times b$ is equal to the area of the parallelogram with
        two of its sides formed by $a$ and $b$. Half of that is the
        area of the triangle formed by $a$ and $b$.
      \item The line going through $a$ and $b$ is $Ax+By=C$ where $A=b_y-a_y$, $B=a_x-b_x$, $C=Aa_x+Ba_y$.
      \item Two lines $A_1x+B_1y=C_1$, $A_2x+B_2y=C_2$ are parallel iff.\ $D=A_1B_2-A_2B_1$ is zero. Otherwise their unique intersection is $(B_2C_1-B_1C_2,A_1C_2-A_2C_1)/D$.
      \item \textbf{Euler's formula:} $V - E + F = 2$
      \item Side lengths $a,b,c$ can form a triangle iff.\ $a+b>c$, $b+c>a$ and $a+c>b$.
      \item Sum of internal angles of a regular convex $n$-gon is $(n-2)\pi$.
      \item \textbf{Law of sines:} $\frac{a}{\sin A} = \frac{b}{\sin B} = \frac{c}{\sin C}$
      \item \textbf{Law of cosines:} $b^2 = a^2 + c^2 - 2ac\cos B$
      \item Internal tangents of circles $(c_1,r_1), (c_2,r_2)$ intersect at $(c_1r_2+c_2r_1)/(r_1+r_2)$, external intersect at $(c_1r_2-c_2r_1)/(r_1+r_2)$.
    \end{itemize}
\section{Other Algorithms}
  \subsection{2SAT}
    \ifverbose
    A fast 2SAT solver.
    \fi
    \code{other/two_sat.cpp}
  \subsection{DPLL Algorithm}
    A SAT solver that can solve a random 1000-variable SAT instance within a second.
    \code{other/dpll.cpp}
  \subsection{Dynamic Convex Hull Trick}
    \code{other/dynamic-ch-trick.cpp}
  \subsection{Stable Marriage}
    \ifverbose
    The Gale-Shapley algorithm for solving the stable marriage problem.
    \fi
    \code{other/stable_marriage.cpp}
  \subsection{Algorithm X}
    \ifverbose
    An implementation of Knuth's Algorithm X, using dancing links. Solves the Exact Cover problem.
    \fi
    \code{other/algorithm_x.cpp}
  \subsection{Matroid Intersection}
    Computes the maximum weight and cardinality intersection of two
    matroids, specified by implementing the required abstract methods, in
    $O(n^3(M_1+M_2))$.
    \code{other/matroid_intersection.cpp}
  \subsection{$n$th Permutation}
    \ifverbose
    A very fast algorithm for computing the $n$th permutation of the list
    $\{0,1,\ldots,k-1\}$.
    \fi
    \code{other/nth_permutation.cpp}
  \subsection{Cycle-Finding}
    \ifverbose
    An implementation of Floyd's Cycle-Finding algorithm.
    \fi
    \code{other/floyds_algorithm.cpp}
  \subsection{Longest Increasing Subsequence}
    \code{other/lis.cpp}
  \subsection{Dates}
    \ifverbose
    Functions to simplify date calculations.
    \fi
    \code{other/dates.cpp}
  \subsection{Simulated Annealing}
    An example use of Simulated Annealing to find a permutation of length $n$
    that maximizes $\sum_{i=1}^{n-1}|p_i - p_{i+1}|$.
    \code{other/simulated_annealing.cpp}
  \subsection{Simplex}
    \code{other/simplex.cpp}
  \subsection{Fast Square Testing}
    An optimized test for square integers.
    \code{tricks/is_square.cpp}
  \subsection{Fast Input Reading}
    If input or output is huge, sometimes it is beneficial to optimize the
    input reading/output writing. This can be achieved by reading all input
    in at once (using fread), and then parsing it manually. Output can also
    be stored in an output buffer and then dumped once in the end (using
    fwrite). A simpler, but still effective, way to achieve speed is to use
    the following input reading method.
    \code{tricks/fast_input.cpp}
  \subsection{128-bit Integer}
    GCC has a 128-bit integer data type named \texttt{\_\_int128}. Useful
    if doing multiplication of 64-bit integers, or something needing a
    little more than 64-bits to represent. There's also
    \texttt{\_\_float128}.
  \subsection{Bit Hacks}
    \code{tricks/snoob.cpp}
\newpage
\section{Other Combinatorics Stuff}
  \begin{tabular}{@{}l|l|l@{}}
  \toprule
  Catalan	&	$C_0=1$, $C_n=\frac{1}{n+1}\binom{2n}{n} = \sum_{i=0}^{n-1}C_iC_{n-i-1} = \frac{4n-2}{n+1}C_{n-1}$  & \\
  Stirling 1st kind & $\left[{0\atop 0}\right]=1$, $\left[{n\atop 0}\right]=\left[{0\atop n}\right]=0$, $\left[{n\atop k}\right]=(n-1)\left[{n-1\atop k}\right]+\left[{n-1\atop k-1}\right]$ & \#perms of $n$ objs with exactly $k$ cycles\\
  Stirling 2nd kind & $\left\{{n\atop 1}\right\}=\left\{{n\atop n}\right\}=1$, $\left\{{n\atop k}\right\} = k \left\{{ n-1 \atop k }\right\} + \left\{{n-1\atop k-1}\right\}$ & \#ways to partition $n$ objs into $k$ nonempty sets\\
  Euler	& $\left \langle {n\atop 0} \right \rangle = \left \langle {n\atop n-1} \right \rangle = 1 $, $\left \langle {n\atop k} \right \rangle = (k+1) \left \langle {n-1\atop {k}} \right \rangle + (n-k)\left \langle {{n-1}\atop {k-1}} \right \rangle$ & \#perms of $n$ objs with exactly $k$ ascents \\
  Euler 2nd Order &  $\left \langle \!\!\left \langle {n\atop k} \right \rangle \!\! \right \rangle = (k+1) \left \langle \!\! \left \langle {{n-1}\atop {k}} \right \rangle \!\! \right \rangle +(2n-k-1)\left \langle \!\! \left \langle {{n-1}\atop {k-1}} \right \rangle  \!\! \right \rangle$ & \#perms of ${1,1,2,2,...,n,n}$ with exactly $k$ ascents \\
  Bell & $B_1 = 1$, $B_n = \sum_{k=0}^{n-1} B_k \binom{n-1}{k} = \sum_{k=0}^n\left\{{n\atop k}\right\}$ & \#partitions of $1..n$ (Stirling 2nd, no limit on k)\\
  \bottomrule
  \end{tabular}

  \vspace{10pt}
  \begin{tabular}{ll}
    \#labeled rooted trees & $n^{n-1}$ \\
    \#labeled unrooted trees & $n^{n-2}$ \\
    \#forests of $k$ rooted trees & $\frac{k}{n}\binom{n}{k}n^{n-k}$ \\
    % Kirchoff's theorem
    $\sum_{i=1}^n i^2 = n(n+1)(2n+1)/6$ & $\sum_{i=1}^n i^3 = n^2(n+1)^2/4$ \\
    $!n = n\times!(n-1)+(-1)^n$ & $!n = (n-1)(!(n-1)+!(n-2))$ \\
    $\sum_{i=1}^n \binom{n}{i} F_i = F_{2n}$ & $\sum_{i} \binom{n-i}{i} = F_{n+1}$ \\
    $\sum_{k=0}^n \binom{k}{m} = \binom{n+1}{m+1}$ & $x^k = \sum_{i=0}^k i!\stirling{k}{i}\binom{x}{i} = \sum_{i=0}^k \left\langle {k \atop i} \right\rangle\binom{x+i}{k}$ \\

    $a\equiv b\pmod{x,y} \Rightarrow a\equiv b\pmod{\lcm(x,y)}$ & $\sum_{d|n} \phi(d) = n$ \\
    $ac\equiv bc\pmod{m} \Rightarrow a\equiv b\pmod{\frac{m}{\gcd(c,m)}}$ & $(\sum_{d|n} \sigma_0(d))^2 = \sum_{d|n} \sigma_0(d)^3$ \\
    $p$ prime $\Leftrightarrow (p-1)!\equiv -1\pmod{p}$ & $\gcd(n^a-1,n^b-1) = n^{\gcd(a,b)}-1$ \\
    $\sigma_x(n) = \prod_{i=0}^{r} \frac{p_i^{(a_i + 1)x} - 1}{p_i^x - 1}$ & $\sigma_0(n) = \prod_{i=0}^r (a_i + 1)$ \\
    $\sum_{k=0}^m (-1)^k \binom{n}{k} = (-1)^m \binom{n-1}{m}$ & \\
    $2^{\omega(n)} = O(\sqrt{n})$ & $\sum_{i=1}^n 2^{\omega(i)} = O(n \log n)$ \\
    % Kinematic equations
    $d = v_i t + \frac{1}{2}at^2$ & $v_f^2 = v_i^2 + 2ad$ \\
    $v_f = v_i + at$ & $d = \frac{v_i + v_f}{2}t$ \\
  \end{tabular}
  \subsection{The Twelvefold Way}
    Putting $n$ balls into $k$ boxes.\\
  \begin{tabular}{@{}c|c|c|c|c|l@{}}
  Balls & same & distinct & same & distinct & \\
  Boxes & same & same & distinct & distinct & Remarks\\
  \hline
    - & $\mathrm{p}_k(n)$ & $\sum_{i=0}^k \stirling{n}{i}$ & $\binom{n+k-1}{k-1}$ & $k^n$ & $\mathrm{p}_k(n)$: \#partitions of $n$ into $\le k$ positive parts \\
    $\mathrm{size}\ge 1$ & $\mathrm{p}(n,k)$ & $\stirling{n}{k}$ & $\binom{n-1}{k-1}$ & $k!\stirling{n}{k}$ & $\mathrm{p}(n,k)$: \#partitions of $n$ into $k$ positive parts \\
    $\mathrm{size}\le 1$ & $[n \le k]$ & $[n \le k]$ & $\binom{k}{n}$ & $n!\binom{k}{n}$ & $[cond]$: $1$ if $cond=true$, else $0$\\
  \bottomrule
  \end{tabular}

\clearpage

\section{Misc}
  \subsection{Debugging Tips}
    \begin{myitemize}
      \item Stack overflow? Recursive DFS on tree that is actually a long path?
      \item Floating-point numbers
        \begin{itemize}
          \item Getting \texttt{NaN}? Make sure \texttt{acos} etc.\ are
            not getting values out of their range (perhaps
            \texttt{1+eps}).
          \item Rounding negative numbers?
          \item Outputting in scientific notation?
        \end{itemize}
      \item Wrong Answer?
        \begin{itemize}
          \item Read the problem statement again!
          \item Are multiple test cases being handled correctly?
            Try repeating the same test case many times.
          \item Integer overflow?
          \item Think very carefully about boundaries of all input parameters
          \item Try out possible edge cases:
            \begin{itemize}
              \item $n=0, n=-1, n=1, n=2^{31}-1$ or $n=-2^{31}$
              \item List is empty, or contains a single element
              \item $n$ is even, $n$ is odd
              \item Graph is empty, or contains a single vertex
              \item Graph is a multigraph (loops or multiple edges)
              \item Polygon is concave or non-simple
            \end{itemize}
          \item Is initial condition wrong for small cases?
          \item Are you sure the algorithm is correct?
          \item Explain your solution to someone.
          \item Are you using any functions that you don't completely understand? Maybe STL functions?
          \item Maybe you (or someone else) should rewrite the solution?
          \item Can the input line be empty?
        \end{itemize}
      \item Run-Time Error?
        \begin{itemize}
          \item Is it actually Memory Limit Exceeded?
        \end{itemize}
    \end{myitemize}

  \subsection{Solution Ideas}
    \begin{myitemize}
      \item Dynamic Programming
        \begin{itemize}
          \item Parsing CFGs: CYK Algorithm
          \item Drop a parameter, recover from others
          \item Swap answer and a parameter
          \item When grouping: try splitting in two
          \item $2^k$ trick
          \item When optimizing
              \begin{itemize}
                \item Convex hull optimization
                  \begin{itemize}
                    \item $\mathrm{dp}[i] = \min_{j<i}\{\mathrm{dp}[j] + b[j] \times a[i]\}$
                    \item $b[j] \geq b[j+1]$
                    \item optionally $a[i] \leq a[i+1]$
                    \item $O(n^2)$ to $O(n)$
                  \end{itemize}
                \item Divide and conquer optimization
                  \begin{itemize}
                    \item $\mathrm{dp}[i][j] = \min_{k<j}\{\mathrm{dp}[i-1][k] + C[k][j]\}$
                    \item $A[i][j] \leq A[i][j+1]$
                    \item $O(kn^2)$ to $O(kn\log{n})$
                    \item sufficient: $C[a][c] + C[b][d] \leq C[a][d] + C[b][c]$, $a\leq b\leq c\leq d$ (QI)
                  \end{itemize}
                \item Knuth optimization
                  \begin{itemize}
                    \item $\mathrm{dp}[i][j] = \min_{i<k<j}\{\mathrm{dp}[i][k] + \mathrm{dp}[k][j] + C[i][j]\}$
                    \item $A[i][j-1] \leq A[i][j] \leq A[i+1][j]$
                    \item $O(n^3)$ to $O(n^2)$
                    \item sufficient: QI and $C[b][c] \leq C[a][d]$, $a\leq b\leq c\leq d$
                  \end{itemize}
              \end{itemize}
        \end{itemize}
      \item Greedy
      \item Randomized
      \item Optimizations
        \begin{itemize}
          \item Use bitset (/64)
          \item Switch order of loops (cache locality)
        \end{itemize}
      \item Process queries offline
        \begin{itemize}
          \item Mo's algorithm
        \end{itemize}
      \item Square-root decomposition
      \item Precomputation
      \item Efficient simulation
        \begin{itemize}
          \item Mo's algorithm
          \item Sqrt decomposition
          \item Store $2^k$ jump pointers
        \end{itemize}
      \item Data structure techniques
        \begin{itemize}
          \item Sqrt buckets
          \item Store $2^k$ jump pointers
          \item $2^k$ merging trick
        \end{itemize}
      \item Counting
        \begin{itemize}
          \item Inclusion-exclusion principle
          \item Generating functions
        \end{itemize}
      \item Graphs
        \begin{itemize}
          \item Can we model the problem as a graph?
          \item Can we use any properties of the graph?
          \item Strongly connected components
          \item Cycles (or odd cycles)
          \item Bipartite (no odd cycles)
            \begin{itemize}
              \item Bipartite matching
              \item Hall's marriage theorem
              \item Stable Marriage
            \end{itemize}
          \item Cut vertex/bridge
          \item Biconnected components
          \item Degrees of vertices (odd/even)
          \item Trees
            \begin{itemize}
              \item Heavy-light decomposition
              \item Centroid decomposition
              \item Least common ancestor
              \item Centers of the tree
            \end{itemize}
          \item Eulerian path/circuit
          \item Chinese postman problem
          \item Topological sort
          \item (Min-Cost) Max Flow
          \item Min Cut
            \begin{itemize}
              \item Maximum Density Subgraph
            \end{itemize}
          \item Huffman Coding
          \item Min-Cost Arborescence
          \item Steiner Tree
          \item Kirchoff's matrix tree theorem
          \item Pr\"ufer sequences
          \item Lov\'asz Toggle
          \item Look at the DFS tree (which has no cross-edges)
          \item Is the graph a DFA or NFA?
              \begin{itemize}
                \item Is it the Synchronizing word problem?
              \end{itemize}
        \end{itemize}
      \item Mathematics
        \begin{itemize}
          \item Is the function multiplicative?
          \item Look for a pattern
          \item Permutations
            \begin{itemize}
              \item Consider the cycles of the permutation
            \end{itemize}
          \item Functions
            \begin{itemize}
              \item Sum of piecewise-linear functions is a piecewise-linear function
              \item Sum of convex (concave) functions is convex (concave)
            \end{itemize}
          \item Modular arithmetic
            \begin{itemize}
              \item Chinese Remainder Theorem
              \item Linear Congruence
            \end{itemize}
          \item Sieve
          \item System of linear equations
          \item Values too big to represent?
            \begin{itemize}
              \item Compute using the logarithm
              \item Divide everything by some large value
            \end{itemize}
          \item Linear programming
            \begin{itemize}
              \item Is the dual problem easier to solve?
            \end{itemize}
          \item Can the problem be modeled as a different combinatorial problem? Does that simplify calculations?
        \end{itemize}
      \item Logic
        \begin{itemize}
          \item 2-SAT
          \item XOR-SAT (Gauss elimination or Bipartite matching)
        \end{itemize}
      \item Meet in the middle
      \item Only work with the smaller half ($\log(n)$)
      \item Strings
        \begin{itemize}
          \item Trie (maybe over something weird, like bits)
          \item Suffix array
          \item Suffix automaton (+DP?)
          \item Aho-Corasick
          \item eerTree
          \item Work with $S+S$
        \end{itemize}
      \item Hashing
      \item Euler tour, tree to array
      \item Segment trees
        \begin{itemize}
          \item Lazy propagation
          \item Persistent
          \item Implicit
          \item Segment tree of X
        \end{itemize}
      \item Geometry
        \begin{itemize}
          \item Minkowski sum (of convex sets)
          \item Rotating calipers
          \item Sweep line (horizontally or vertically?)
          \item Sweep angle
          \item Convex hull
        \end{itemize}
      \item Fix a parameter (possibly the answer).
      \item Are there few distinct values?
      \item Binary search
      \item Sliding Window (+ Monotonic Queue)
      \item Computing a Convolution? Fast Fourier Transform
      \item Computing a 2D Convolution? FFT on each row, and then on each column
      \item Exact Cover (+ Algorithm X)
      \item Cycle-Finding
      \item What is the smallest set of values that identify the solution? The cycle structure of the permutation? The powers of primes in the factorization?
      \item Look at the complement problem
        \begin{itemize}
          \item Minimize something instead of maximizing
        \end{itemize}
      \item Immediately enforce necessary conditions. (All values greater than 0? Initialize them all to 1)
      \item Add large constant to negative numbers to make them positive
      \item Counting/Bucket sort
    \end{myitemize}

\section{Formulas}

  % \item Number of permutations of length $n$ that have no fixed
  %     points (derangements): $D_0 = 1, D_1 = 0, D_n = (n - 1)(D_{n-1}
  %     + D_{n-2})$
  % \item Number of permutations of length $n$ that have exactly $k$
  %     fixed points: $\binom{n}{k} D_{n-k}$


  \begin{itemize}[leftmargin=*]
    \item \textbf{Legendre symbol:} $\left(\frac{a}{b}\right) = a^{(b-1)/2} \pmod{b}$, $b$ odd prime.
    \item \textbf{Heron's formula:} A triangle with side lengths
      $a,b,c$ has area $\sqrt{s(s-a)(s-b)(s-c)}$ where $s =
      \frac{a+b+c}{2}$.
    \item \textbf{Pick's theorem:} A polygon on an integer grid
      strictly containing $i$ lattice points and having $b$ lattice
      points on the boundary has area $i + \frac{b}{2} - 1$. (Nothing
      similar in higher dimensions)
    \item \textbf{Euler's totient:} The number of integers less than
      $n$ that are coprime to $n$ are $n\prod_{p|n}\left(1 - \frac{1}{p}\right)$
      where each $p$ is a distinct prime factor of $n$.
    \item \textbf{König's theorem:} In any bipartite graph $G=(L\cup R,E)$, the number
      of edges in a maximum matching is equal to the number of
      vertices in a minimum vertex cover. Let $U$ be the set of
      unmatched vertices in $L$, and $Z$ be the set of vertices that
      are either in $U$ or are connected to $U$ by an alternating
      path. Then $K=(L\setminus Z)\cup(R\cap Z)$ is the minimum
      vertex cover.
    \item A minumum Steiner tree for $n$ vertices requires at most $n-2$ additional Steiner vertices.
    \item The number of vertices of a graph is equal to its minimum
      vertex cover number plus the size of a maximum independent set.
    \item \textbf{Lagrange polynomial} through points $(x_0,y_0),\ldots,(x_k,y_k)$ is $L(x) = \sum_{j=0}^k y_j \prod_{\shortstack{$\scriptscriptstyle 0\leq m \leq k$ \\ $\scriptscriptstyle m\neq j$}} \frac{x-x_m}{x_j - x_m}$
    \item \textbf{Hook length formula:} If $\lambda$ is a Young diagram and $h_{\lambda}(i,j)$ is the hook-length of cell $(i,j)$, then then the number of Young tableux $d_{\lambda} = n!/\prod h_{\lambda}(i,j)$.
    \item \textbf{Möbius inversion formula:} If $f(n) = \sum_{d|n} g(d)$, then $g(n) = \sum_{d|n} \mu(d) f(n/d)$. If $f(n) = \sum_{m=1}^n g(\lfloor n/m\rfloor)$, then $g(n) = \sum_{m=1}^n \mu(m)f(\lfloor\frac{n}{m}\rfloor)$.
    \item \#primitive pythagorean triples with hypotenuse $<n$ approx $n/(2\pi)$.
    \item \textbf{Frobenius Number:} largest number which can't be
      expressed as a linear combination of numbers $a_1,\ldots,a_n$
      with non-negative coefficients. $g(a_1,a_2) = a_1a_2-a_1-a_2$,
      $N(a_1,a_2)=(a_1-1)(a_2-1)/2$. $g(d\cdot a_1,d\cdot a_2,a_3) =
      d\cdot g(a_1,a_2,a_3) + a_3(d-1)$. An integer $x>\left(\max_i
      a_i\right)^2$ can be expressed in such a way iff.\ $x\ |\
      \mathrm{gcd}(a_1,\ldots,a_n)$.
  \end{itemize}

  \subsection{Physics}
    \begin{itemize}
      \item \textbf{Snell's law:} $\frac{\sin\theta_1}{v_1} = \frac{\sin\theta_2}{v_2}$
    \end{itemize}

  \subsection{Markov Chains}
    A Markov Chain can be represented as a weighted directed graph of
    states, where the weight of an edge represents the probability of
    transitioning over that edge in one timestep. Let $P^{(m)} = (p^{(m)}_{ij})$
    be the probability matrix of transitioning from state $i$ to state $j$
    in $m$ timesteps, and note that $P^{(1)}$ is the adjacency matrix of
    the graph. \textbf{Chapman-Kolmogorov:} $p^{(m+n)}_{ij} = \sum_{k}
    p^{(m)}_{ik} p^{(n)}_{kj}$. It follows that $P^{(m+n)} =
    P^{(m)}P^{(n)}$ and $P^{(m)} = P^m$. If $p^{(0)}$ is the initial
    probability distribution (a vector), then $p^{(0)}P^{(m)}$ is the
    probability distribution after $m$ timesteps.

    The return times of a state $i$ is $R_i = \{m\ |\ p^{(m)}_{ii} > 0 \}$,
    and $i$ is \textit{aperiodic} if $\gcd(R_i) = 1$. A MC is aperiodic if
    any of its vertices is aperiodic. A MC is \textit{irreducible} if the
    corresponding graph is strongly connected.

    A distribution $\pi$ is stationary if $\pi P = \pi$. If MC is
    irreducible then $\pi_i = 1/\mathbb{E}[T_i]$, where $T_i$ is the
    expected time between two visits at $i$. $\pi_j/\pi_i$ is the expected
    number of visits at $j$ in between two consecutive visits at $i$. A MC
    is \textit{ergodic} if $\lim_{m\to\infty} p^{(0)} P^{m} = \pi$. A MC is
    ergodic iff.\ it is irreducible and aperiodic.

    A MC for a random walk in an undirected weighted graph (unweighted
    graph can be made weighted by adding $1$-weights) has $p_{uv} =
    w_{uv}/\sum_{x} w_{ux}$. If the graph is connected, then $\pi_u =
    \sum_{x} w_{ux} / \sum_{v}\sum_{x} w_{vx}$. Such a random walk is
    aperiodic iff.\ the graph is not bipartite.

    An \textit{absorbing} MC is of the form $P = \left(\begin{matrix} Q & R
    \\ 0 & I_r \end{matrix}\right)$. Let $N = \sum_{m=0}^\infty Q^m = (I_t
    - Q)^{-1}$. Then, if starting in state $i$, the expected number of
    steps till absorption is the $i$-th entry in $N1$. If starting in state
    $i$, the probability of being absorbed in state $j$ is the $(i,j)$-th
    entry of $NR$.

    Many problems on MC can be formulated in terms of a system of
    recurrence relations, and then solved using Gaussian elimination.

  \subsection{Burnside's Lemma}
    Let $G$ be a finite group that acts on a set $X$. For each $g$ in $G$
    let $X^g$ denote the set of elements in $X$ that are fixed by $g$. Then
    the number of orbits \[ |X/G| = \frac{1}{|G|} \sum_{g\in G} |X^g| \]

    \[
        Z(S_n) = \frac{1}{n} \sum_{l=1}^n a_l Z(S_{n-l})
    \]

  \subsection{Bézout's identity}
    If $(x,y)$ is any solution to $ax+by=d$ (e.g.\ found by the Extended
    Euclidean Algorithm), then all solutions are given by \[
    \left(x+k\frac{b}{\gcd(a,b)}, y-k\frac{a}{\gcd(a,b)}\right) \]

  \subsection{Misc}
    \subsubsection{Determinants and PM}
      \begin{align*}
        det(A) &= \sum_{\sigma \in S_n}\text{sgn}(\sigma)\prod_{i = 1}^n a_{i,\sigma(i)}\\
        perm(A) &= \sum_{\sigma \in S_n} \prod_{i = 1}^n a_{i,\sigma(i)}\\
        pf(A) &= \frac{1}{2^nn!}\sum_{\sigma \in S_{2n}} \text{sgn}(\sigma)\prod_{i = 1}^n a_{\sigma(2i-1),\sigma(2i)}\\ &= \sum_{M \in \text{PM}(n)} \text{sgn}(M) \prod_{(i,j) \in M} a_{i,j}
      \end{align*}

    \subsubsection{BEST Theorem}
      Count directed Eulerian cycles. Number of OST given by
      Kirchoff's Theorem (remove r/c with root) $\#\textsc{OST}(G,r)
      \cdot \prod_{v}(d_v-1)!$

    \subsubsection{Primitive Roots}
      Only exists when $n$ is $2, 4, p^k, 2p^k$, where $p$ odd prime. Assume
      $n$ prime. Number of primitive roots $\phi(\phi(n))$
      Let $g$ be primitive root. All primitive roots are of the form $g^k$
      where $k,\phi(p)$ are coprime.\\ $k$-roots:
      $g^{i \cdot \phi(n) / k}$ for $0 \leq i < k$

    \subsubsection{Sum of primes} For any multiplicative $f$:
      \[
          S(n,p) = S(n, p-1) - f(p) \cdot (S(n/p,p-1) - S(p-1,p-1))
      \]

    \subsubsection{Floor}
      \begin{align*}
          &\left\lfloor \left\lfloor x/y \right\rfloor / z \right\rfloor = \left\lfloor x / (yz) \right\rfloor \\
          &x \% y = x - y \left\lfloor x / y \right\rfloor
      \end{align*}


\iffalse
  \clearpage
  \section*{Practice Contest Checklist}
    \begin{itemize}
      \item How many operations per second? Compare to local machine.
      \item What is the stack size?
      \item How to use printf/scanf with long long/long double?
      \item Are \texttt{\_{}\_{}int128} and \texttt{\_{}\_{}float128} available?
      \item Does MLE give RTE or MLE as a verdict? What about stack overflow?
      \item What is \texttt{RAND\_{}MAX}?
      \item How does the judge handle extra spaces (or missing newlines) in the output?
      \item Look at documentation for programming languages.
      \item Try different programming languages: C++, Java and Python.
      \item Try the submit script.
      \item Try local programs: i?python[23], factor.
      \item Try submitting with \texttt{assert(false)} and \texttt{assert(true)}.
      \item Return-value from \texttt{main}.
      \item Look for directory with sample test cases.
      \item Make sure printing works.

      \item Remove this page from the notebook.
    \end{itemize}
\fi


\end{multicols*}

\end{document}
