\documentclass[8pt,a4paper,landscape,oneside]{amsart}
\usepackage{amsmath, amsthm, amssymb, amsfonts}
\usepackage[T1]{fontenc}
\usepackage[utf8]{inputenc}
\usepackage{booktabs}
\usepackage{caption}
\usepackage{color}
\usepackage{fancyhdr}
\usepackage{float}
\usepackage{fullpage}
%\usepackage{geometry}
% \usepackage[top=0pt, bottom=1cm, left=0.3cm, right=0.3cm]{geometry}
\usepackage[top=3pt, bottom=1cm, left=0.3cm, right=0.3cm]{geometry}
\usepackage{graphicx}
% \usepackage{listings}
\usepackage{subcaption}
\usepackage[scaled]{beramono}
\usepackage{titling}
\usepackage{datetime}
\usepackage{enumitem}
\usepackage{multicol}
\usepackage{bookmark}

\newcommand{\subtitle}[1]{%
  \posttitle{%
    \par\end{center}
    \begin{center}\large#1\end{center}
    \vskip0.1em\vspace{-1em}}%
}

% Minted
\usepackage{minted}
\newcommand{\code}[1]{\inputminted[fontsize=\normalsize,baselinestretch=1]{cpp}{_code/#1}}
\newcommand{\bashcode}[1]{\inputminted{bash}{_code/#1}}
\newcommand{\regcode}[1]{\inputminted{cpp}{code/#1}}

% Header/Footer
% \geometry{includeheadfoot}
%\fancyhf{}
\pagestyle{fancy}
\lhead{Ateneo de Manila University}
\rhead{\thepage}
\cfoot{}
\setlength{\headheight}{15.2pt}
\setlength{\droptitle}{-20pt}
\posttitle{\par\end{center}}
\renewcommand{\headrulewidth}{0.4pt}
\renewcommand{\footrulewidth}{0.4pt}

% Math and bit operators
\DeclareMathOperator{\lcm}{lcm}
\newcommand*\BitAnd{\mathrel{\&}}
\newcommand*\BitOr{\mathrel{|}}
\newcommand*\ShiftLeft{\ll}
\newcommand*\ShiftRight{\gg}
\newcommand*\BitNeg{\ensuremath{\mathord{\sim}}}
\DeclareRobustCommand{\stirling}{\genfrac\{\}{0pt}{}}

\newenvironment{myitemize}
{ \begin{itemize}[leftmargin=.5cm]
    \setlength{\itemsep}{0pt}
    \setlength{\parskip}{0pt}
    \setlength{\parsep}{0pt}     }
{ \end{itemize}                  }

% Title/Author
\title{$L^3$}
\subtitle{Team Notebook}
\date{\ddmmyyyydate{\today{}}}

% Output Verbosity
\newif\ifverbose
\verbosetrue
% \verbosefalse

\begin{document}

\begin{multicols*}{3}
\maketitle
\thispagestyle{fancy}
\vspace{-3em}
% \addtocontents{toc}{\protect\enlargethispage{\baselineskip}}
\tableofcontents

% \clearpage

\section{Code Templates}
	\code{header.cpp}
\section{Data Structures}
	\subsection{Fenwick Tree}
		\subsubsection{Point Queries}
			\code{data-structures/fenwick.cpp}
		\subsubsection{Range Queries}
	\subsection{Mergesort Tree}
	\subsection{Segment Tree}
		\subsubsection{Recursive Segment Tree (Point-update)}
			\code{data-structures/segtree_rec.cpp}
		\subsubsection{Iterative Segment Tree (Point-update \and operation can be non-commutative)}
			\code{data-structures/segtree_iter.cpp}
		\subsubsection{Lazy Segment Tree (Range-update)}
			\code{data-structures/segtree_lazy.cpp}
		\subsubsection{Persistent Segmentr Tree (Point-update)}
			\code{data-structures/segtree_persistent.cpp}
	\subsection{Sparse Table}
	\subsection{Sqrt Decomposition}
	\subsection{Treap}
		\subsubsection{Explicit Treap}
		\subsubsection{Implicit Treap}
			\code{data-structures/treap_implicit.cpp}
		\subsubsection{Persistent Treap}
	\subsection{Ordered Statistics Tree}
	\subsection{Union Find}
		\code{data-structures/union_find.cpp}
\section{Graphs}
	Using adjacency list:
	\code{graphs/graph_template_adjlist.cpp}
	Using adjacency matrix:
	\code{graphs/graph_template_adjmat.cpp}
	Using edge list:
	\code{graphs/graph_template_edgelist.cpp}
	\subsection{Single-Source Shortest Paths}
		\subsubsection{Dijkstra}
			\code{graphs/dijkstra.cpp}
		\subsubsection{Bellman-Ford}
			\code{graphs/bellman_ford.cpp}
	\subsection{All-Pairs Shortest Paths}
		\subsubsection{Floyd-Washall}
			\code{graphs/floyd_warshall.cpp}
	\subsection{Strongly Connected Components}
		\subsubsection{Kosaraju}
	\subsection{Cut Points and Bridges}
	\subsection{Biconnected Components}
		\subsubsection{Bridge Tree}
		\subsubsection{Block-Cut Tree}
	\subsection{Minimum Spanning Tree}
		\subsubsection{Kruskal}
		\subsubsection{Prim}
	\subsection{Topological Sorting}
	\subsection{Euler Path}
	\subsection{Bipartite Matching}
		\subsubsection{Alternating Paths Algorithm}
		\subsubsection{Hopcroft-Karp Algorithm}
	\subsection{Maximum Flow}
		\subsubsection{Edmonds-Karp}
			\code{graphs/edmonds_karp.cpp}
		\subsubsection{Dinic}
			\code{graphs/dinic.cpp}
  \subsection{All-pairs Maximum Flow}
    \subsubsection{Gomory-Hu}
  \subsection{Heavy Light Decomposition}
    \code{graphs/heavy_light_decomposition.cpp}
	\subsection{Centroid Decomposition}
	\subsection{Least Common Ancestor}
		\subsubsection{Binary Lifting}
		\subsubsection{Tarjan's Offline Algorithm}
\section{Strings}
	\subsection{Z-algorithm}
	\subsection{Trie}
	\subsection{Hashing}
\section{Dynamic Programming}
	\subsection{Longest Common Subsequence}
	\subsection{Longest Increasing Subsequence}
	\subsection{Traveling Salesman}
\section{Mathematics}
	\subsection{Special Data Types}
		\subsubsection{Fraction}
		\subsubsection{BigInteger}
		\subsubsection{Matrix}
		\subsubsection{Dates}
	\subsection{Binomial Coefficients}
	\subsection{Euclidean Algorithm}
	\subsection{Primality Test}
		\subsubsection{Optimized Brute Force}
		\subsubsection{Miller-Rabin}
		\subsubsection{Pollard's Rho Algorithm}
	\subsection{Sieve}
		\subsubsection{Sieve of Eratosthenes}
		\subsubsection{Divisor Sieve (Modified Sieve of Eratosthenes)}
		\subsubsection{Phi Sieve}
	\subsection{Phi Function}
	\subsection{Modular Exponentiation}
	\subsection{Modular Multiplicative Inverse}
	\subsection{Chinese Remainder Theorem}
	\subsection{Numeric Integration (Simpson's Rule)}
	\subsection{Fast Fourier Transform}
	\subsection{Josephus Problem}
	\subsection{Number of Integer Points Below a Line}
\section{Geometry}
	\subsection{Primitives}
	\subsection{Lines}
	\subsection{Circles}
	\subsection{Polygons}
	\subsection{Convex Hull (Graham's Scan)}
	\subsection{Closest Pair of Points}
	\subsection{Rectilinear Minimum Spanning Tree}
\section{Other Algorithms}
	\subsection{Coordinate Compression}
	\subsection{2SAT}
	\subsection{Nth Permutation}
	\subsection{Floyd's Cycle-Finding}
	\subsection{Simulated Annealing}
	\subsection{Hexagonal Grid Algorithms}

\clearpage

\input{useful_info}

\end{multicols*}

\end{document}
